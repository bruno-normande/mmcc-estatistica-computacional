\documentclass[12pt]{beamer}
\usepackage{beamerthemeCopenhagen}
\usecolortheme[rgb={0.18,0.3,0.3}]{structure}
%\usepackage[T1]{fontenc}
\usepackage[utf8]{inputenc}
%\usepackage[latin1]{inputenc}
%\usepackage[brazil]{babel}
\usepackage{amsmath}
\usepackage{amsfonts}
\usepackage{amsthm}
\usepackage{pgf,tikz}
\usepackage{bbm}

\usepackage{graphicx}
\usepackage{url,color,ae}
\usepackage{listings,color,upquote}
\usepackage{verbatim}
\usepackage{multirow}
\usepackage{booktabs}

\usetikzlibrary{arrows}

\newcommand{\caixa}{$\Box$}
\newtheorem{defi}{Definição}
\newtheorem{ex}{Exemplo}
\newtheorem{teo}{Teorema}
\newtheorem{cor}{Corolário}
\newtheorem{lema}{Lema}
\newtheorem{dist}{Distribuição}
\newtheorem{emv}{Máxima verossimilhança}
\newtheorem{mo1}{Primeiro momento}
\newtheorem{mo2}{Segundo momento}
\newtheorem{prova}{Demonstração}
\newtheorem{BiNeg}{Definição}
\setbeamertemplate{navigation symbols}{}

\institute[IC-UFAL]{Universidade Federal de Alagoas\\
Instituto de Computação}
\title{Comparação de estimadores pelo método Monte Carlo das distribuições: $\mathcal{G}_{\rm I}^ {0}$, Uniforme, Binomial negativa, Gamma e Exponencial}
\author{Antônio Marcos Larangeiras\and Alisson Nascimento\and
  Paulo\and Bruno Normande\and Juliana Leal}
\date{{\tiny Maceió-AL, Dezembro de 2013}}
\begin{document}
\begin{frame}
\maketitle
\begin{picture}(0,0)(0,-50)
\includegraphics[scale=0.11]{IC.png}
\end{picture}
\begin{picture}(0,0)(-250,-40)
\includegraphics[scale=0.13]{ufal.png}
\end{picture}
\end{frame} 

%=========================
\section{Introdução}
%=========================


%================================================================================
\begin{frame}
\frametitle{Introdução}
\begin{itemize}
\item Calcular o estimador por máxima verossimilhança;

\bigskip

\item Calcular estimador pelo primeiro momento;

\bigskip

\item Calcular estimador pelo segundo momento;

\bigskip

\item Utilizar o Método Monte Carlo força bruta.

\end{itemize}
\end{frame}
%===============================================================================


%========================
\section{Distribuições}
%========================

%==============================================================================================
\subsection{Distribuição Uniforme}
\begin{frame}
\frametitle{Distribui\c c\~ ao Uniforme}
Admitimos que em um dado problema, números reais cobrem de forma uniforme um segmento de reta $[a--b]$ de tal maneira que quando se observa qualquer subintervalo contenha o mesmo número de pontos, e portanto , equiprovável.

Sua função de distribuição é dada por:

\begin{equation}
\label{eq:U_d}
f(x) =
\left \{
\begin{array}{cc}
\frac{1}{ a + b } & a \leqslant x \leqslant b\\
0 & \textnormal{caso contrário.} \\
\end{array}
\right.
\end{equation}

\end{frame}

\begin{frame}
\frametitle{Distribui\c c\~ ao Uniforme}

Aplicando a função \ref{eq:U_d} a uma variável aleatória $X\sim U(x; 0,\theta)$, com $\theta>0$. 

Sua função de distribuição é esta:

\begin{displaymath}
f(x) =
\left \{
\begin{array}{cc}
\frac{1}{ \theta } & 0 \leqslant x \leqslant \theta\\
0 & \textnormal{caso contrário.} \\
\end{array}
\right.
\end{displaymath}


\end{frame}

%========================================================================================================================================================
\begin{frame}
\frametitle{Distribuição $U(x; 0,\theta)$}

A esperânça matemática da distribuição $U(x; 0,\theta)$:

\begin{equation}
\label{eq:U_esp}
E(X)=\dfrac{\theta}{2}
\end{equation}\pause

A variância da distribuição $U(x; 0,\theta)$:

\begin{equation}
\label{eq:U_var}
Var(X)=\dfrac{\theta^{2}}{12}
\end{equation}

\end{frame}

%====================
\subsubsection{Estimadores}
%====================
\begin{frame}
\frametitle{Estimadores da Distribuição $U(x; 0,\theta)$}
Pela Máxima Veressimilhança:
\small \begin{equation}
\label{eq:U_MV}
\hat{\theta}={\dfrac{1}{\theta ^ n }}{ \mathbbm{1}_{\max{\{X_1,\dots,X_n\}\leqslant\theta}}},
\textnormal{portanto } \hat{\theta} = {\max{\{X_1,\dots,X_n\}\leqslant\theta}} 
\end{equation} \pause
Pelo momento primeiro momento:
\begin{eqnarray}
\label{eq:U_m_p2}
\hat{\theta}_{2}=2\overline{X}
\end{eqnarray}
\end{frame}

\begin{frame}
\frametitle{Estimadores da Distribuição $U(x; 0,\theta)$}
Pelo segundo momento:
\begin{equation}
\label{eq:U_m_p3}
\hat{\theta}_{3} = \sqrt{{{{3} \over {n}} {\sum_{i=1}^{n} X_{i}^{2}}}}
\end{equation}

\end{frame}


%===================
\subsubsection{Gráficos}
%=========================================================================================================================================================
\begin{frame}
\frametitle{Gráficos da distribuição $U(x; 0,\theta)$}

\end{frame}
%=========================================================================================================================================================





%==============================================================================================
\subsection{Distribuição $\mathcal{G}_{\rm I}^{0}$}

\begin{frame}
\frametitle{Distribui\c c\~ ao $\mathcal{G}_{\rm I}^{0}$}

Denotaremos esta distribuição por $Z\sim\mathcal{G}_{\rm I}^{0}(z; \alpha,1,1)$, com $-\alpha, z>0$. Sua densidade é

\begin{displaymath}
f_{Z}=(z;\alpha,1,1)=\dfrac{\Gamma(L-\alpha)}{\Gamma(-\alpha)(1+z)^{1-\alpha}}, \mbox{com} -\alpha \ \mbox{e}\ z>0.
\end{displaymath}


\end{frame}

%========================================================================================================================================================
\begin{frame}
\frametitle{Distribui\c c\~ ao $\mathcal{G}_{\rm I}^{0}$}

A esperança matemática da distribuição $\mathcal{G}_{\rm I}^{0}$:

\begin{equation}
\label{eq:Gi0_esp}
E(Z)=\dfrac{\gamma}{\alpha-1}
\end{equation}\pause

A variância da distribuição $\mathcal{G}_{\rm I}^{0}$:

\begin{equation}
\label{eq:Gi0_var}
Var(Z)=\dfrac{1}{(\alpha-1)^2}+\dfrac{1}{(\alpha-2)(\alpha-1)^2}+\dfrac{1}{\alpha-1}
\end{equation}

\end{frame}

%====================
\subsubsection{Estimadores}
%====================
\begin{frame}
\frametitle{Estimadores da Distribuição $\mathcal{G}_{\rm I}^{0}$}
Pela Máxima Veressimilhança:
\begin{equation}
\label{eq:Gi0_MV}
\hat{\alpha}=\dfrac{n}{\displaystyle\sum_{i=1}^{n}\ln(1+z_i)}
\end{equation} \pause
Pelo momento central de ordem 2:
\begin{eqnarray}
\label{eq:Gi0_m_p2}
\hat{\alpha}_2=\overline{Z}
\end{eqnarray}
\end{frame}

\begin{frame}
\frametitle{Estimadores da Distribuição $\mathcal{G}_{\rm I}^{0}$}
Pelo segundo momento:
\begin{equation}
\label{eq:Gi0_m_p3}
\hat{\alpha}_3 = \frac{\frac{3}{k}\displaystyle\sum_{i=1}^{k}Z_{i}^{2}+2+\sqrt{\frac{1}{k^{2}}\displaystyle\sum_{i=1}^{k}Z_{i}^{4}+4}}{\frac{2}{k}\displaystyle\sum_{i=1}^{k}Z_{i}^{2}}
\end{equation}

\end{frame}


%===================
\subsubsection{Gráficos}
%=========================================================================================================================================================
\begin{frame}
\frametitle{Gráficos da distribuição $\mathcal{G}_{\rm I}^{0}$}

\end{frame}
%=========================================================================================================================================================




\subsection{Distribuição Binomial Negativa}
\begin{frame}
\frametitle{Distribuição Binomial Negativa}
Considere a situação de observar um evento dicotômico $X_{i}$ independentes e identicamente distribuídos segundo uma lei de Bernoulli de probabilidade $p$. Suponha que se registre $X$, o número de ensaios até obter exatamente $k$ sucessos. \pause

\begin{dist}[aaa] 
Seja uma variavel aleatoria que fornece o numero de ensaios até o k-simo sucesso. 
%Assim, $ X$ tem uma distribuição binomial negativa com parâmetro $p \in (0,1)$, 
%se sua função de probabilidade é dada por:

%\begin{displaymath}
%P_{r}(X=x)=\left\{ \begin{array}{ll}
%\binom{x-1}{k-1} \cdot p^{k} \cdot (1-p)^{x-k} & \textrm{se  x=k,k+1, \ldots.}\\
%0 & \textrm{, caso contrário}
%\end{array} \right.
%\end{displaymath}
Usualmente utilizá-se a notação $X \sim BN(p,k)$.
\end{dist}
\end{frame}
%==========================================================================
\subsection{Distribuição $\Gamma(w;k,\theta)$}
\begin{frame}
  \frametitle{Distribuição $\Gamma(w;k,\theta)$}
  
  A Distribuição Gamma é caracterizada por dois valores, denominados
  \textit{shape} ($k$) e \textit{scale} ($\theta$). Ela possui a
  seguinte função densidade probabilidade:

  \begin{displaymath}
    f(w;k,\theta) = \frac{w^{k-1}e^{-\frac{w}{\theta}}}{\theta^k\Gamma(w)}
  \end{displaymath}
  
  para, $ w > 0 $ e $ k,\theta > 0 $ 
\end{frame}

%==========================================================================
\begin{frame}
  \frametitle{Distribuição $\Gamma(w;k,\theta)$}

  Esperança da distribuição $\Gamma(w;k,\theta)$:

  \begin{equation}
    \label{eq:G_esp}
     E[W] = k\theta
  \end{equation}\pause
  
  A variância da distribuição $\Gamma(w;k,\theta)$:
  
  \begin{equation}
    \label{eq:G_var}
      Var[W] = k\theta^2
  \end{equation}

  Em nosso estudo usaremos sempre $\theta = 1$ para simplicidade dos cálculos.
\end{frame}

% ====================
\subsubsection{Estimadores}
% ====================
\begin{frame}
  \frametitle{Estimadores da Distribuição $\Gamma(w;k,1)$}
  
  Para estimar pela Máxima Verossimilhança é preciso encontrar o
  máximo da função log-verossimilhança de $\Gamma(w;k,1)$

  \small \begin{align*}
%    \label{eq:Gamma_MV}
    log(p(W|k,1)) &= n(k-1)\overline{log(x)} - n log(\Gamma(k)) - n k
    log(\overline{x}) + \\
    & n k log(a) - n k
    \intertext{Que podemos resolver numericamente iterando sobre k em:}
%    log(p(W|k,1)) &\approx c_0 + c_1a + c_2log(a)
    \frac{1}{k} &= \frac{1}{k_0} + \frac{\overline{log(x)} -
      log(\overline{x}) + log(k_0) - \psi(k_0)}{k_0^2 (\frac{1}{k_0} -
      \psi'(k_0))}
  \end{align*}
\end{frame}

\begin{frame}
  \frametitle{Estimadores da Distribuição $\Gamma(w;k,1)$}
  Quando $k \approx k_0$ então encontramos o estimador $\hat{k}$  

  \small \begin{align*}
%    \label{eq:Gamma_MV}
    \intertext{Como k inicial podemos usar a seguinte aproximação}
    \hat{k} &= \frac{0.5}{log(\overline{x}) - \overline{log(x)}}
  \end{align*}
\end{frame}

\begin{frame}
  \frametitle{Estimadores da Distribuição $\Gamma(w;k,1)$}
  
  Estimador pelo primeiro momento:
  
    \begin{align*}
%      \label{eq:G_m_p2}
      E[W] &= k\theta \\
      \hat{k}_1 &= \frac{1}{n} \sum_{i=1}^nw_i 
    \end{align*}
\end{frame}

\begin{frame}
  \frametitle{Estimadores da Distribuição $\Gamma(w;k,1)$}
  
  Estimador pelo segundo momento central:
  
    \begin{align*}
%      \label{eq:G_m_p3}
      Var[W] &= k\theta^2 \\
      \hat{k}_2^0 &= Var(w)  
    \end{align*}
\end{frame}


%===================
\subsubsection{Gráficos}
%=========================================================================================================================================================
\begin{frame}
\frametitle{Gráficos da distribuição $U(x; 0,\theta)$}

\end{frame}
%=========================================================================================================================================================




%==============================================================================================
\subsection{Distribuição Exponencial}
\begin{frame}
\frametitle{Distribuição Exponencial}
A distribuição exponencial é frequentemente usada em estudos de confiabilidade como sendo um modelo para o tempo até a falha de um equipamento. Essa distribuição funciona como o inverso da distribuição Poisson. Enquanto a Poisson estima a quantidade de eventos em um intervalo, a exponencial analisa um intervalo ou espaço para a ocorrência de um evento. 

Sua função densidade é dada por:

\begin{equation}
 F(v) = 1 - e^{-\lambda v}, \textnormal{para } v \geq 0.
\end{equation}

\end{frame}

%========================================================================================================================================================
\begin{frame}
\frametitle{Distribuição $Exp(\lambda)$}

A esperança matemática da distribuição $Exp(\lambda)$:

\begin{equation}
\label{eq:Exp_esp}
E(V)=1/\lambda
\end{equation}\pause

A variância da distribuição $Exp(\lambda)$:

\begin{equation}
\label{eq:Esp_var}
Var(V)=1/\lambda^2
\end{equation}

\end{frame}

%====================
\subsubsection{Estimadores}
%====================
\begin{frame}
\frametitle{Estimadores da Distribuição $Exp(\lambda)$}
Pela Máxima Veressimilhança:
\small \begin{equation}
\label{eq:Exp_MV}
\hat{\lambda}=1/\bar{V}
\end{equation} \pause
Pelo momento primeiro momento:
\begin{eqnarray}
\label{eq:Exp_m_p2}
\hat{\lambda}_{2}=1/\lambda
\end{eqnarray}
\end{frame}

\begin{frame}
\frametitle{Estimadores da Distribuição $Exp(\lambda)$}
Pelo segundo momento:
\begin{equation}
\label{eq:Exp_m_p3}
\hat{\lambda}_{3} = 2/\lambda^2
\end{equation}

\end{frame}


%===================
\subsubsection{Gráficos}
%=========================================================================================================================================================
\begin{frame}
\frametitle{Gráficos da distribuição $Exp(\lambda)$}

\end{frame}
%=========================================================================================================================================================





%===================
\section{Resultados}
%===================

%================================================================================
%\input{../unforme/result_uniforme}
\subsection{Distribuição Binomial Negaiva}
\begin{frame}
\frametitle{Distribuição $BN(p,k)$}
\tiny
\begin{table}[h]
\caption{Viés dos estimadores $\hat{p_{1}}$ e $\tilde{p_{1}}$.}
\label{tab:p1Vies}
\centering
\begin{tabular}{rcccc}
\toprule
\multicolumn{5}{c}{Comparação dos viés dos Estimadores $\hat{p_{1}}$ e $\tilde{p_{1}}$}\\
$n$ & $p$ & $B(\hat{p_{1}})$ & $B(\tilde{p_{1}})$ & $|B(\hat{p_{1}})|>|B(\tilde{p_{1}})|$ \\
\midrule
50 & 0.1 & 0.01158347 & 0.01108683 & TRUE \\
100 & 0.1 & 0.01147125 & 0.01124116 & TRUE \\
150 & 0.1 & 0.01129354 & 0.01113929 & TRUE \\
100000 & 0.1 & 0.01110255 & 0.01110307 & FALSE \\
\midrule
50 & 0.2 & 0.05069611 & 0.04952346 & TRUE \\
100 & 0.2 & 0.05113692 & 0.05050091 & TRUE \\
150 & 0.2 & 0.05021439 & 0.04981072 & TRUE\\
100000 & 0.2 & 0.05000023 & 0.05000045 & FALSE \\
\midrule
50 & 0.3 & 0.13303390 & 0.13062580 & TRUE \\
100 & 0.3 & 0.12839159 & 0.12709425 & TRUE \\
150 & 0.3 & 0.12910398 & 0.12821978 & TRUE \\
100000 & 0.3 & 0.12858837 & 0.12858547 & TRUE \\
\bottomrule
\end{tabular}
\end{table}
\end{frame}

\begin{frame}
\frametitle{Distribuição $BN(p,k)$}

\tiny
\begin{table}[h]
\caption{EQM dos estimadores $\hat{p_{1}}$ e $\tilde{p_{1}}$.}
\label{tab:p1EQM}
\centering
\begin{tabular}{rcccc}
\toprule
\multicolumn{5}{c}{Comparação dos EQM do Estimador $\hat{p_{1}}$ e $\tilde{p_{1}}$}\\
$n$ & $p$ & $EQM(\hat{p_{1}})$ & $EQM(\tilde{p_{1}})$ & $EQM(\hat{p_{1}})>EQM(\tilde{p_{1}})$ \\
\midrule
50 & 0.1 & 0.0001897657 & 0.0001792158 & TRUE \\
100 & 0.1 & 0.0001601663 & 0.0001552416 & TRUE \\
150 & 0.1 & 0.0001461537 & 0.0001429903 & TRUE \\
100000 & 0.1 & 0.0001232947 & 0.0001233068 & FALSE \\
\midrule
50 & 0.2 & 0.0028992563 & 0.0027791147 & TRUE \\
100 & 0.2 & 0.0027753925 & 0.0027122546 & TRUE \\
150 & 0.2 & 0.0026332882 & 0.0025933471 & TRUE\\
100000 & 0.2 & 0.0025001601 & 0.0025001831 & FALSE \\
\midrule
50 & 0.3 & 0.0188113709 & 0.0181701887 & TRUE \\
100 & 0.3 & 0.0170132255 & 0.0166791609 & TRUE \\
150 & 0.3 & 0.0170230361 & 0.0167979328 & TRUE \\
100000 & 0.3 & 0.0165355188 & 0.0165347806 & TRUE \\
\bottomrule
\end{tabular}
\end{table}
\end{frame}


\begin{frame}
\frametitle{Distribuição $BN(p,k)$}
\tiny
\begin{table}[h]
\caption{Comparação dos estimadores $\hat{p_{2}}$ e $\tilde{p_{2}}$.}
\label{tab:p2}
\centering
\begin{tabular}{rccc}
\toprule
\multicolumn{4}{c}{Comparação dos Estimadores $\hat{p_{2}}$ e $\tilde{p_{2}}$}\\
$n$ & $p$ & $|B(\hat{p_{2}})|>|B(\tilde{p_{2}})|$ & $EQM(\hat{p_{2}})>EQM(\tilde{p_{2}})$ \\
\midrule
50 & 0.1 & TRUE & FALSE \\
100 & 0.1 & TRUE & FALSE \\
150 & 0.1 & TRUE & FALSE \\
100000 & 0.1 & TRUE & FALSE \\
\midrule
50 & 0.2 & TRUE & TRUE \\
100 & 0.2 & TRUE & FALSE \\
150 & 0.2 & FALSE & FALSE\\
100000 & 0.2 & TRUE & FALSE \\
\midrule
50 & 0.3 & TRUE & FALSE \\
100 & 0.3 & TRUE & FALSE \\
150 & 0.3 & FALSE & FALSE \\
100000 & 0.3 & FALSE & FALSE \\
\bottomrule
\end{tabular}
\end{table}
\end{frame}

\begin{frame}
\frametitle{Distribuição $BN(p,k)$}
\tiny
\begin{table}[h]
\caption{Comparação dos estimadores $\hat{p_{3}}$ e $\tilde{p_{3}}$.}
\label{tab:p2}
\centering
\begin{tabular}{rccc}
\toprule
\multicolumn{4}{c}{Comparação dos Estimadores $\hat{p_{3}}$ e $\tilde{p_{3}}$}\\
$n$ & $p$ & $|B(\hat{p_{3}})|>|B(\tilde{p_{3}})|$ & $EQM(\hat{p_{3}})>EQM(\tilde{p_{3}})$ \\
\midrule
50 & 0.1 & TRUE & TRUE \\
100 & 0.1 & TRUE & TRUE \\
150 & 0.1 & TRUE & TRUE \\
100000 & 0.1 & TRUE & TRUE \\
\midrule
50 & 0.2 & TRUE & TRUE \\
100 & 0.2 & TRUE & TRUE \\
150 & 0.2 & TRUE & TRUE \\
100000 & 0.2 & TRUE & TRUE \\
\midrule
50 & 0.3 & TRUE & TRUE \\
100 & 0.3 & TRUE & TRUE \\
150 & 0.3 & TRUE & TRUE \\
100000 & 0.3 & TRUE & TRUE \\
\bottomrule
\end{tabular}
\end{table}
\end{frame}
\input{../exponencial/result_exp}
%================================================================================

%=====================
\section{Conclusões}
%=====================

%================================================================================
\begin{frame}
\frametitle{Conclusões}

\end{frame}
%===============================================================================

\section{Referências Bibliográficas}

\begin{frame}
\frametitle{Referências Bibliográficas}


\bibliographystyle{plain}

\bibliography{Larangeiras}


\end{frame}
\end{document}