\subsection{Distribuição Binomial Negativa}
\begin{frame}
\frametitle{Distribuição Binomial Negativa}
Considere a situação de observar um evento dicotômico $Y_{i}$ independentes e identicamente distribuídos segundo uma lei de Bernoulli de probabilidade $p$. Suponha que se registre $Y$, o número de ensaios até obter exatamente $k$ sucessos. 

\end{frame}

\begin{frame}
\frametitle{Distribuição Binomial Negativa}

Seja uma variável aleatória que fornece o numero de ensaios até o k-ésimo sucesso. 
Assim, $Y$ tem uma distribuição binomial negativa com parâmetro $p \in (0,1)$, 
se sua função de probabilidade é dada por:

\begin{displaymath}
P_{r}(Y=y)=\left\{ \begin{array}{ll}
\binom{y-1}{k-1} \cdot p^{k} \cdot (1-p)^{y-k} & \textrm{se  y=k,k+1, \ldots.}\\
0 & \textrm{, caso contrário}
\end{array} \right.
\end{displaymath}


\end{frame}

%========================================================================================================================================================
\begin{frame}
\frametitle{Distribui\c c\~ ao $BN(p,k)$}

Usualmente sua função de probabilidade denota-se: $Y \sim BN(p,k)$.\pause

A esperança matemática da distribuição binomial negativa:

\begin{equation}
\label{eq:esp}
E(X) = {\frac{k}{p}}
\end{equation}\pause

A variância da distribuição binomail negativa:

\begin{equation}
\label{eq:var}
Var(X) = {\frac{k(1-p)}{p^{2}}}
\end{equation}

\end{frame}

%====================
\subsubsection{Estimadores}
%====================
\begin{frame}
\frametitle{Estimadores da Distribuição $BN(p,k)$}
Pela Máxima Veressimilhança:
\begin{equation}
\label{eq:BN_MV}
{\hat{p}} = {\frac{n k}{\sum_{i=1}^{n}{Y_{i}}}}
\end{equation} \pause
Pelo momento central de ordem 2:
\begin{eqnarray}
\label{eq:BN_m_p2}
{\hat{p}_{2}} = {\frac{-k + \sqrt{k^{2}+4k{Var(X_{i})}}}{2{Var(X_{i})}}}
\end{eqnarray}
\end{frame}

\begin{frame}
\frametitle{Estimadores da Distribuição $BN(p,k)$}
Pelo segundo momento:
\begin{equation}
\label{eq:BN_m_p3}
{\widehat{p}_{3}} = {\frac{{-k} + {\sqrt{k^{2} + 4 E(X_{i}^{2})(k^{2}+k)}}}{2E(X_{i}^{2})}}
\end{equation}

\end{frame}


%===================
\subsubsection{Gráficos}
%=========================================================================================================================================================
\begin{frame}
\frametitle{Gráficos da distribuição $BN(p,k)$}

\end{frame}
%=========================================================================================================================================================

