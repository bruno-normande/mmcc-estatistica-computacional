\subsection{Distribuição Binomial Negativa}
\begin{frame}
\frametitle{Distribuição Binomial Negativa}
Considere a situação de observar um evento dicotômico $X_{i}$ independentes e identicamente distribuídos segundo uma lei de Bernoulli de probabilidade $p$. Suponha que se registre $X$, o número de ensaios até obter exatamente $k$ sucessos. \pause

\begin{dist}[aaa] 
Seja uma variavel aleatoria que fornece o numero de ensaios até o k-simo sucesso. 
%Assim, $ X$ tem uma distribuição binomial negativa com parâmetro $p \in (0,1)$, 
%se sua função de probabilidade é dada por:

%\begin{displaymath}
%P_{r}(X=x)=\left\{ \begin{array}{ll}
%\binom{x-1}{k-1} \cdot p^{k} \cdot (1-p)^{x-k} & \textrm{se  x=k,k+1, \ldots.}\\
%0 & \textrm{, caso contrário}
%\end{array} \right.
%\end{displaymath}
Usualmente utilizá-se a notação $X \sim BN(p,k)$.
\end{dist}
\end{frame}